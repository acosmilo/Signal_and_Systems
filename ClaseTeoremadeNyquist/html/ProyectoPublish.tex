
% This LaTeX was auto-generated from MATLAB code.
% To make changes, update the MATLAB code and republish this document.

\documentclass{article}
\usepackage{graphicx}
\usepackage{color}

\sloppy
\definecolor{lightgray}{gray}{0.5}
\setlength{\parindent}{0pt}

\begin{document}

    
    
\section*{Deber Tercer Parcial}

\begin{par}
Grabar una senal de voz con frecuencia de muestreo de 16kHz guardar esa senal en un archivo y graficarla en el tiempo y en la frecuencia. A la misma senal cambiar de frecuencia de muestreo a 8kHz, 4kHz, 2kHz y 1kHz. A esas senales guardarlas y graficarlas en tiempo y frecuencia.
\end{par} \vspace{1em}

\subsection*{Contents}

\begin{itemize}
\setlength{\itemsep}{-1ex}
   \item Resolucion
   \item Parametros iniciales
   \item Grabacion
   \item Grafica senal a 16kHz
\end{itemize}


\subsection*{Resolucion}

\begin{par}
Se cierra figuras, se limpia variables y consola.
\end{par} \vspace{1em}
\begin{verbatim}
close all
clear all
clc
\end{verbatim}


\subsection*{Parametros iniciales}

\begin{par}
Se asignan valores a 3 parametros
\end{par} \vspace{1em}
\begin{enumerate}
\setlength{\itemsep}{-1ex}
   \item Fs : la frecuencia de muestreo.
   \item Bits : numero de bits, con los que se otendra la grabacion.
   \item Canal : cuantos canales se audio de entrada se usaran.
\end{enumerate}
\begin{verbatim}
Fs = 16000; % Frecuencia de muestreo 16kHz.
Bits = 16;
Canal = 1; % Un unico canal audio mono.
\end{verbatim}


\subsection*{Grabacion}

\begin{par}
Usando la funcion \textit{audiorecorder()} se graba la informacion del audio en un objeto llamado \textbf{recObj}. Usando la funcion \textit{recordblocking()} ordenamos que grabe unicamente 5 segundos.
\end{par} \vspace{1em}
\begin{verbatim}
disp('Comienza grabacion REC'); % Mensaje de inicio en consola.
recObj = audiorecorder (Fs, Bits, Canal);
recordblocking (recObj, 5);
disp('Finaliza grabacion STOP'); % Mensaje de finalizado en consola.
% Usando _getaudiodata()_ convertimos la informacion grabada en *recObj* en
% un vector y se la almacena en la variable audio.
audio=getaudiodata(recObj);
% Se guarda el audio a Fs=16kHz con formato .wav
audiowrite ('audio16kHz.wav', audio , Fs);
audio16=audio;
\end{verbatim}

        \color{lightgray} \begin{verbatim}Comienza grabacion REC
Finaliza grabacion STOP
\end{verbatim} \color{black}
    

\subsection*{Grafica senal a 16kHz}

\begin{par}
Se crea un eje temporal  \textit{t} que permita ver la senal en 5 segundos, por ello al vector que va de 0 hasta 79999 se lo divide para Fs = 16000 para que el vector valla desde 0 hasta 4.9, entonces el audio estaria en funcion del tiempo.
\end{par} \vspace{1em}
\begin{verbatim}
t=(0:length(audio)-1)/Fs;

f=Fs/length(audio)*(0:length(audio)-1);
Y=real(fft(audio)/length(audio));

figure
subplot(1,2,1)
plot (t,audio,'Color',[0,0,1]);
xlim([0 5])
ylim([-1.25 1.25])
xlabel('Tiempo[s]');
ylabel('Amplitud')
title('Senal a 16k[Hz]')

subplot(1,2,2)
plot (f,Y,'Color',[0,0.5,1])
xlim([0 8000])
ylim([0 0.005])
xlabel('Frecuencia 16k[Hz]');
ylabel('|Y(f)|')
title('Espectro de Frecuencia')


%%%%%%%%%%%%%%primero a 8kHz

audiowrite ('audio8kHz.wav', audio, (Fs/2));

audio(2:2:end)=[];
audio8=audio;
t1=(0:length(audio)-1)/(Fs/2);

f1=(Fs/2)/length(audio)*(0:length(audio)-1);
Y=real(fft(audio)/length(audio));

figure
subplot(1,2,1)
plot(t1,audio,'Color',[0.5,0.5,1])
xlim([0 5])
ylim([-1.25 1.25])
xlabel('Tiempo[s]');
ylabel('Amplitud')
title('Senal a 8k[Hz]')

subplot(1,2,2)
plot (f1,Y,'Color',[0.5,1,1])
xlim([0 4000])
ylim([0 0.005])
xlabel('Frecuencia 8k[Hz]');
ylabel('|Y(f)|')
title('Espectro de Frecuencia')

%%%%%%%%%%%%%%primero a 4kHz

audiowrite ('audio4kHz.wav', audio, (Fs/4));

audio(2:2:end)=[];
audio4=audio;
t2=(0:length(audio)-1)/(Fs/4);

f2=(Fs/4)/length(audio)*(0:length(audio)-1);
Y=real(fft(audio)/length(audio));

figure
subplot(1,2,1)
plot(t2,audio,'Color',[1,0,1])
xlim([0 5])
ylim([-1.25 1.25])
xlabel('Tiempo[s]');
ylabel('Amplitud')
title('Senal a 4k[Hz]')

subplot(1,2,2)
plot (f2,Y,'Color',[1 0.4 0.6])
xlim([0 2000])
ylim([0 0.005])
xlabel('Frecuencia[Hz]');
ylabel('|Y(f)|')
title('Espectro de Frecuencia')


%%%%%%%%%%%%%%primero a 2kHz

audiowrite ('audio2kHz.wav', audio, (Fs/8));

audio(2:2:end)=[];
audio2=audio;
t3=(0:length(audio)-1)/(Fs/8);

f3=(Fs/8)/length(audio)*(0:length(audio)-1);
Y=real(fft(audio)/length(audio));

figure
subplot(1,2,1)
plot(t3,audio,'Color',[0.3 1 0.7])
xlim([0 5])
ylim([-1.25 1.25])
xlabel('Tiempo[s]');
ylabel('Amplitud')
title('Senal a 2k[Hz]')

subplot(1,2,2)
plot (f3,Y,'Color',[0.5 1 0])
xlim([0 1000])
ylim([0 0.005])
xlabel('Frecuencia[Hz]');
ylabel('|Y(f)|')
title('Espectro de Frecuencia')

%%%%%%%%%%%%%%primero a 1kHz

audiowrite ('audio1kHz.wav', audio, (Fs/16));

audio(2:2:end)=[];
audio1=audio;
t4=(0:length(audio)-1)/(Fs/16);

f4=(Fs/16)/length(audio)*(0:length(audio)-1);
Y=abs(fft(audio)/length(audio));

figure
subplot(1,2,1)
plot(t4,audio,'Color',[1 0 0.5])
xlim([0 5])
ylim([-1.25 1.25])
xlabel('Tiempo[s]');
ylabel('Amplitud')
title('Senal a 2k[Hz]')

subplot(1,2,2)
plot (f4,Y,'Color',[1 0.5 0])
xlim([0 500])
ylim([0 0.005])
xlabel('Frecuencia[Hz]');
ylabel('|Y(f)|')
title('Espectro de Frecuencia')
\end{verbatim}

\includegraphics [width=4in]{ProyectoPublish_01.eps}

\includegraphics [width=4in]{ProyectoPublish_02.eps}

\includegraphics [width=4in]{ProyectoPublish_03.eps}

\includegraphics [width=4in]{ProyectoPublish_04.eps}

\includegraphics [width=4in]{ProyectoPublish_05.eps}



\end{document}
    
